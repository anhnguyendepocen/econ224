\documentclass[11pt, letterpaper]{article}
\usepackage{geometry}
\geometry{margin=1in} 
\usepackage{setspace}
\linespread{1}
\usepackage{hyperref}
\usepackage{totcount}
\usepackage{termcal}
\usepackage{enumerate}
\usepackage{fancybox}
\usepackage{amsmath, amssymb}


% Some useful commands (our classes always meet either on Monday and Wednesday 
% or on Tuesday and Thursday)
\newcommand{\MWClass}{%
\calday[Monday]{\classday} % Monday
\skipday % Tuesday (no class)
\calday[Wednesday]{\classday} % Wednesday
\skipday % Thursday (no class)
\skipday % Friday 
\skipday\skipday % weekend (no class)
}

\newcommand{\TRClass}{%
\skipday % Monday (no class)
\calday[Tuesday]{\classday} % Tuesday
\skipday % Wednesday (no class)
\calday[Thursday]{\classday} % Thursday
\skipday % Friday 
\skipday\skipday % weekend (no class)
}

\newcommand{\Holiday}[2]{%
\options{#1}{\noclassday}
\caltext{#1}{#2}
}

\begin{document}


\thispagestyle{plain}

\begin{center}
\Large
\sc
Statistical Learning and Causal Inference for Economics\\
\large
Economics 224\\
\large
Fall 2018
\end{center}



\normalsize

\noindent \textbf{Course Instructor:} Francis DiTraglia \\
Office: MCNB 535\\
Office Hours: M 2--4pm 

\medskip


%\noindent \textbf{Recitation Instructors:}
%
%\medskip
%\noindent
%
%\begin{tabular}{llll}
%  & Assa Cohen & Philippe Goulet Coulombe & Gabrielle Vasey\\
%Office:& MCNB 421 & MCNB 342 & MCNB 420 \\ 
%Office Hours:& T 6:30--8:30pm & W 1--3pm & R 1--3pm  
%\end{tabular}
%
%\medskip
% 
\noindent \textbf{Lecture Time and Location:} TR 9-10:30AM, location TBA



\medskip
 
\noindent \textbf{Course Website:} Course materials will be posted at \url{http://ditraglia.com/econ224}.
You can view your grades and log-on to the course discussion forum, Piazza, at \url{https://canvas.upenn.edu}.

\medskip

\noindent \textbf{Email Policy:}
Please direct all written communication concerning Econ 224 to the course discussion forum, Piazza, rather than to the instructor or RI's personal email accounts.
For questions about course material and logistics, please make your post visible to the whole class, so that others can benefit from your question and our response.
(You are welcome to post anonymously.)
For personal issues, use the private messaging feature to communicate directly with the instructor. 

\medskip



\noindent \textbf{Course Description:} 


\medskip


\noindent \textbf{Prerequisites:} 
The prerequisite for this course is Econ 103 (Statistics for Economists) or comparable coursework from the Statistics Department, e.g.\ STAT 430 and 431. 
You are expected to be conversant with the basics of probability and statistics, including confidence intervals, hypothesis testing, and simple linear regression.





\medskip

\noindent \textbf{Required Texts:} There are three required texts for this course:
\begin{itemize}
  \item ``Mastering `Metrics'' (MM) by Angrist \& Pischke
  \item ``An Introduction to Statistical Learning'' (ISL) by James, Witten, Hastie, \& Tibshirani: \url{http://www-bcf.usc.edu/~gareth/ISL/}
  \item ``R for Data Science'' (RDS) by Wickham \& Grolemund: \url{http://r4ds.had.co.nz}
\end{itemize}
Note that ISL and RDS are freely available from their respective authors at the listed urls. 
If you prefer a physical copy, printed versions are available at the Penn bookstore and on Amazon.
While MM is not available for free, it is quite inexpensive: around \$30 new or \$20 used on Amazon. 
Because MM contains many equations 
 

\medskip


\noindent \textbf{Required Software:} 
We will use the statistical package R via a front-end called RStudio throughout the course. 
Both R and RStudio are free and open source. Installation instructions appear on the last page of this syllabus.
RStudio is also available in the Undergraduate Data Analysis Lab (UDAL) in McNeil rooms 104 and 108--9. 

\medskip

\noindent \textbf{Departmental Course Policies: } 
All Economics Department course policies are in force in Econ 103 even if not explicitly listed on this syllabus. 
See: \url{http://economics.sas.upenn.edu/undergraduate-program/course-information/guidelines/policies} for full details. 


\bigskip


\noindent \textbf{Academic Integrity: } 
All suspected violations of the code of academic integrity as set forth in the Pennbook will be reported to the Office of Student Conduct. 
Confirmed violations will result in a failing grade for the course. 

\medskip

\noindent \textbf{Piazza:} 
We will be using an online discussion forum called Piazza, accessible via \href{http://upenn.instructure.com}{Canvas}, for all written communication in this course.
We will use Piazza to make course announcements, answer questions about course material and respond to private messages from individual students regarding personal issues.
By asking your question and getting an answer on Piazza, you create a positive externality: other students benefit from your questions and you benefit from theirs.
You can even post anonymously if asking questions publicly makes you uncomfortable.
The instructor and RIs will actively moderate Piazza both to answer questions and approve (or correct) answers written by your fellow-students.
As mentioned above under ``Email Policy,'' all written communication for Econ 103 should be directed to Piazza, not to the instructor's personal email accounts.

\medskip

\noindent \textbf{Homework:} 



\section*{Assignments and Grading}

Grades for this course will be determined based on ???
Specifically,
	\begin{equation*}
	\begin{split}
		\mbox{Overall Score} = (30\% \times \mbox{Quizzes})  + (20\% \times \mbox{Midterm 1}) + (20\% \times \mbox{Midterm 2}) + (30\% \times \mbox{Final}).
	\end{split}
	\end{equation*}

\medskip 

\noindent \textbf{Course Curve:}

\medskip


\noindent \textbf{Quizzes:} 

\medskip

\noindent \textbf{Exams:} 

\medskip

\noindent \textbf{Regrade Requests:}
Exam regrade requests must be made in writing within a week of receiving your graded exam. 
As we re-grade the entire exam, your score could rise or fall. 
You may not discuss your answers with an RI or the instructor before submitting a regrade request. 


\section*{Installing R and RStudio} First, download and install R from \url{http://cran.r-project.org/}. Second, download and install RStudio by visiting \url{http://rstudio.org/download/desktop} and clicking the link listed under ``Recommended for Your System.'' 

\newpage


%NOTE: don't use leading zeros in dates! In other words, use 1/1/2014 rather than 01/01/2014

\begin{center}
\footnotesize
\begin{calendar}{8/27/2018}{15} %Date of Monday in first week of classes, NOT the date of the first class!
\setlength{\calboxdepth}{.25in}
\TRClass

% schedule
\caltexton{1}{Course Outline/Policies, Pre-test\\ Lab 1: R Review, RMarkdown Overview} 

\caltextnext{Intro.\ to Prediction and Classification (ISL Ch.\ 2)\\ Lab 2: Graduation Rates at US Colleges}
% College dataset: ISL Chapter 2, exercise 8

\caltextnext{Randomized Trials, Potential Outcomes (MM Ch.\ 1)\\ Lab 3: Racial Discrimination in the Labor Market}
% Pischke Problem Set: Chapter 1 Questions 2 and 3. (Bertrand \& Mullainathan experiment vs.\ CPS data)
% Homework: STAR dataset (Imai Exercise 2.8.1)

\caltextnext{Linear Regression for Prediction I (ISL 3.1--3.2)\\ Lab 4: Predicting Election Outcomes}
% Various examples from Imai Chapter 4 on prediction

\caltextnext{Linear Regression for Prediction II (ISL 3.3--3.5)\\ Lab 5:  Crime and House Prices in Boston}
% Boston House Prices data (from MASS), covered in ISL Chapter 3, also covered in Exercise 10 from Chapter 2

\caltextnext{Regression and Causality I (MM 2.1--2.2)\\ Lab 6: Data Visualization with \texttt{ggplot}}
% From Angrist's problem set for MM. The point is to get a baseline for when we later do the regression discontinuity analysis of the same data

\caltextnext{Regression and Causality II (MM 2.3, Appendix)\\ Lab 7: Class Size and Student Achievement}
% From Angrist's problem set for MM. The point is to get a baseline for when we later do the regression discontinuity analysis of the same data


\caltextnext{Logistic Regression (ISL 4.1--4.3)\\ Lab 8: Contaminated Drinking Water in Bangladesh}
% The example is from Gelman and Hill. Could also try the Rodent example from the problems to accompany that chapter.
% Homework: weekly stock market returns: problem 10 from ISL Chapter 4

\caltextnext{Using Bayes' Theorem for Classification (ISL 4.4)\\ Lab 9:  Predicting Individual Ethnicity}
% Also simple linear discriminant analysis (p = 2)
% Imai instructor's repo: Probability chapter

\caltextnext{Instrumental Variables I (MM 3.1--3.2) \\ Lab 10: Transforming Data with \texttt{dplyr}}

\caltextnext{Instrumental Variables II (MM 3.3) \\ Lab 11: Institutions and Economic Development}
% Angrist has problem set questions comparing AJR and Sachs
% Pischke has an even better one!

\caltextnext{Cross-validation and the Bootstrap (ISL Ch.\ 5) \\ Lab 12: Predicting Credit Card Default}
% One of the examples from ISL Chapter 5

\caltextnext{Model Selection (ISL 6.1) \\ Lab 13: Baseball Player Salaries}

\caltextnext{Shrinkage Methods (ISL 6.2, 6.4) \\ Lab 14: Predicting College Applications}
% Example 9 from Chapter 6 of ISL

\caltextnext{Tree-Based Methods (ISL 8.1 [1 \& 3], 8.2 [1 \& 2])\\ Lab 15: American Housing Survey}

\caltextnext{Prediction Competition I}

\caltextnext{Prediction Competition II}

\caltextnext{Regression Discontinuity I (MM 4.1)\\ Lab 15:}

\caltextnext{Regression Discontinuity II (MM 4.2)\\ Lab 16:}

\caltextnext{Differences-in-Differences I (MM 5.1) \\ Lab 17:}

\caltextnext{Differences-in-Differences II (MM 5.2)\\ Lab 18:}

% This one should follow the diff-in-diff lecture
\caltextnext{Clustering Methods (ISL 10.3) \\ Lab 19: The Fox News Effect - Media Bias and Voting}
% QSS instructor repo: Chapter 4 Prediction 




% Holidays
\Holiday{10/4/2018}{\textbf{Fall Break -- No Class}}
\Holiday{11/22/2018}{\textbf{Thanksgiving -- No Class}}
%\Holiday{12/11/2018}{\textbf{Reading Day -- No Class}}
%\Holiday{12/13/2018}{\textbf{Exam Period -- No Class}}
\end{calendar}
\end{center}


% Some other interesting exercises: the Fox News Effect (QSS instructors material)

\end{document}
